⌘יחידה*{נהלי הקורס} 
      נהלי הקורס המלאים והמחייבים זמינים לכל דיכפין, ואילו מבארים היטב עניינים כגון דרישות קדם, דרישות צמודות, תרגילי הבית, מבנה המבחן, וכיוצ"ב.
      אִם לְךָ אוֹבֶה
      שעות הקבלה של המרצה הן בתיאום של 24 שעות מראש, באמצעות מערכת הזמנות אלקטרונית. שעות הקבלה מיועדות לטיפול בבעיות אישיות, ולסיוע אישי

      בנוסף, ניתן לפנות גם בדוא"ל למרצה בשאלות אישיות, ויהיו אלו הקנטרניות ביותר\footnote{דרך משל,  "שותפי העבדקן צבע זקנו בשני, צבע המזכיר לי אנשובי, אשר איני יכול לשאת את ריחו. מה עלי לעשות?" היא לגיטימית.}.
      \begin{itemize}

     \item 
      פניות בדוא"ל אל המרצה בנושא החומר של הקורס, להבהרת קושיה מתחום החומר הנלמד, הסבר נוסף, חידוד הסבר קיים, וכו', תזכה להתעלמות או למתן התשובה: אני משיב לשאלות כלליות אך ורק בשעות הקבלה (או, בר"ת, אם לך אוב"ה \footnote{ ויפטיר החרזן הלץ: "אִם לְךָ אוֹבֶה, יעזרני יהוה"}). שאלות מסוג אלו, יש להם ענין לקהל הלומדים כולו ומקומן ברשות הרבים,  בקבוצת הפייסבוק של הקורס, או באתר השו"ת.  פניה פומבית נותנת מידע ועוזרת לסטודנטים אחרים, ומאפשרת גם תגובה של סטודנטים אחרים בקורס. 
      
      המבקש הסבר אישי המיוחד לו, יטרח ויגיע באופן אישי לשעות הקבלה שם יקבל הסבר כזה בסבר פנים יפות.
    \item 
     פניות בנושאים מנהלתיים, כגון שעות ההרצאה, מבחנים, וכיוצ"ב, צריכות לבוא מהאחראי האקדמי מטעם אגודת הסטודנטים, המרכז אותן, שוקל את שיקולי כלל הסטודנטים, מעריך את חשיבותן, ומצרפן לפניות אחרות בהתאם לצורך. רוב רובן של פניותיו של הרכז האקדמי נענות בחיוב, וכולן זוכות להתייחסות רצינית מאוד.
\end{itemize}
⌘יחידה*{ספרי לימוד וחומר עזר}
      קורס זה נבנה על יסודה של המהדורה השניה של ספרו של David Watt נושא השם "Programming Languages Concepts and Paradigms". אבל, במקומות רבים מאוד נרחיב מעבר לכתוב בספר. ספר הלימוד הזה מתבסס על שלוש שפות תכנות:
      \begin{itemize}
      \item Pascal
      \item ֵML
      \item Ada
     \end{itemize} 
      הספר מעט מיושן, ולעיתים משתמש במונחים באורח אידיוסינקרטי מעט. אמנם הספר יצא לאור במהדורה שלישית, אך לא נשתמש בה כיוון שהיא זונחת את שְׂפַת התכנות ML לטובת שְׂפַת התכנות Haskell.
      ניתן למצוא את השקפים של המהדורה השלישית באינטרנט יחד עם חומר נוסף באתר המחבר:
      http://www.dcs.gla.ac.uk/~daw/books/PLDC/
      ניתן גם לרכוש את הספר כאן:

      http://preview.tinyurl.com/watt2
⌘יחידה*{מבחן}
      המבחן יערך בחומר סגור, יכלול 10 שאלות, אשר משקל כל אחת מהן 10 נק'. קושיין של השאלות והזמן הנדרש לפתירתן אינו זהה. דף השער של המבחן בקורס מצוי כאן. האתר http://safot.cs.technion.ac.il מרכז שאלות ממבחנים קודמים ונועד לאימון בפתרון שאלות ממבחנים קודמים (בעיקר), לקראת הכנה למבחן.
      במבחן תופיע שאלה אחת לפחות מבין כל אחת מארבעת הקטיגוריות הבאות:
      \begin{enumerate}
      \item תרגילי בית של הסמסטר
      \item מבחנים משנים אחרונות (ללא הגדרה מדוייקת של טווח)
      \item תרגילים מחוברת השקפים
      \item שאלות מאתר ההכנה למבחן: 
    ⌘שי{ 
      http://safot.cs.technion.ac.il
      }
      שאלות מסוג זה תיקראנה שאלות "ממוחזרות". כיוון שהקטיגוריות אינן זרות, התנאי יכול (אך לא חייב) להתקיים באורח חופף, ועל כן לא מובטח כי במבחן תהיינה 4 שאלות ממוחזרות.   בכל זאת,  המטרה הכללית היא כ-40\% מהניקוד במבחן יהיה מבוסס על שאלות ממוחזרות.
      \end{enumerate}
