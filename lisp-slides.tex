%\documentclass[a4paper,12pt,reqno]{book}
%\usepackage[notheorems,ignorenonframetext]{beamerarticle}
\documentclass[a4paper]{article}
\usepackage{beamerarticle}
%\documentclass[ignorenonframetext,red]{beamer}
\mode<article>{\usepackage{fullpage}}
\mode<presentation>{\usetheme{Berlin}}
% everyone:
\usepackage[english]{babel}
\usepackage{pgf}
\pgfdeclareimage[height=1cm]{myimage}{filename}
\begin{document}
\section{Introduction}
This is the introduction text. This text is not shown in the
presentation, but will be part of the article.
\begin{frame}
\begin{figure}
% In the article, this is a floating figure,
% In the presentation, this figure is shown in the first frame
\pgfuseimage{myimage}
\end{figure}
\end{frame}
This text is once more not shown in the presentation.
\section{Main Part}
While this text is not shown in the presentation, the section command
also applies to the presentation.
We can add a subsection that is only part of the article like this:
\subsection<article>{Article-Only Section}
With some more text.
\begin{frame}
This text is part both of the article and of the presentation.
\begin{itemize}
\item This stuff is also shown in both version.
\item This too.
\only<article>{\item This particular item is only part
of the article version.}
\item<presentation:only@0> This text is also only part of the article.
\end{itemize}
\end{frame}
\end{document}
\documentclass[ignorenonframetext,notheorems]{beamer}
\usepackage{00}
\begin{document}
\begin{frame}
\frametitle{A frame title}
\begin{itemize}
\item<1-> You can use overlay specifications.
\item<2-> This is useful.
\end{itemize}
\end{frame}

\input lisp.tex
\end{document}
