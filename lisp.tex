§ ביטויים סימבוליים 
בהינתן אלפאבית~$Σ$, נגדיר גם את~$S(Σ)$, קבוצת ה"ביטויים הסימבוליים",(%
\RL{Symbolic Expression, או בקיצור S-Exrepssions}%
)
מעל Σ. ביטוי סימבולי כזה יכול להיות אטומי, ואז הוא חייב להיות אות מ-Σ. ביטוי
סימבולי שאינו אטומי הוא בהכרח זוג סדור של שני ביטויים סימבוליים אחרים.
הזוג הסדור נכתב עטוף בזוג סוגריים זוויתיים~$⟨$ ו~$⟩$, ושני הביטויים הסימבוליים
שבו המופרדים בסימן הנקודה (".").
כמה ביטויים סימבוליים מעל~$Σ=❴a,b,c❵$
הם \[
  a,b,⟨a.b⟩,⟨c.⟨b.a⟩⟩,⟨⟨a.b⟩.⟨a.c⟩⟩∈S❨❴a,b,c❵❩.
\] האיור הבא מתאר את המבנה של ביטויים סימבוליים כעצים בינאריים.
\begin{figure}
  \centering
  \begin{tikzpicture}
    \tikzstyle{data}=[shape=circle,
    draw, align=center,text centered,
    top color=blue!50, bottom color=blue!10]
    \tikzstyle{cons}=[rectangle split,rectangle split parts=2,rectangle split horizontal,draw,text centered,top color=gray!50,bottom color=gray!10,minimum width=3em,minimum height=2em]
    \usetikzlibrary{trees,chains}
    \begin{scope}[start chain=growing right,minimum size=2em]
      \node [data,on chain]{$a$};
      \node [data,on chain]{$b$};
      \node [cons,on chain]{$∙$\nodepart{two}$∙$}
      child {node[data]{$a$}}
      child {node[data]{$b$}};
      \node [cons,on chain,xshift=4ex]{$∙$\nodepart{two}~$∙$}
      child {node[data]{$c$}}
      child {node[cons]{$∙$\nodepart{two}~$∙$}
          child {node[data]{$b$}}
          child {node[data]{$a$}}
        }
      ;
      \node [cons,on chain,xshift=14ex]
      {$∙$\nodepart{two}~$∙$}
      child {node [cons]
            {$∙$\nodepart{two}~$∙$}
          child {node[data]{a}}
          child {node[data]{b}}}
      child {node [cons]
            {$∙$\nodepart{two}~$∙$}
          child {node[data]{a}}
          child {node[data]{c}}}
      ;
    \end{scope}
  \end{tikzpicture}
\end{figure}

לעומת זאת,~$⟨a.b.c⟩$ אינו שייך ל~$S(Σ)$ משום שהוא שלשה סדורה ולא זוג סדור,
ואילו~$⟨a⟨b⟨c⟩⟩⟩$ אינו שייך לקבוצה, משום שהוא אינו עונה על הדרישה שבין פריטים
ימצא סימן הנקודה.

ביטויים סימבוליים הם למעשה עצים בינאריים כאשר צומת פנימית של עץ כזה אינו נושא
מידע, ואילו עלה מכיל מילה מתוך~$Σ$.

לשם ההגדרה הפורמלית נוסיף לאלפאבית~$Σ$ עוד שלושה סימנים שאינם נמצאים בתוכו:
"$.$" (סימן הנקודה), ו "$⟨$" ו "$⟩$".

\begin{Definition}[ביטויים סימבוליים מעל אלפאבית]
  בהנתן אלפאבית~$Σ$ אזי,~$S(Σ)$, קבוצת הביטויים הסימבוליים מעל~$Σ$
  מוגדרת באמצעות הבנאי הנולארי (כלומר איברים אטומיים)
  \begin{equation*}
    \infer{σ∈S(Σ)}{σ∈Σ^*}
  \end{equation*}
  והבנאי הבינארי:
  \begin{equation*}
    \infer{⟨τ₁.τ₂⟩∈S(Σ)}{τ₁∈S(Σ) &τ₂∈S(Σ)}
  \end{equation*}
\end{Definition}

הביטויים הסימבוליים מעל~Σ הם למעשה שפה פורמלית מעל האלפאבית
המורחב:~$Σ∪❴.,⟨,⟩❵$, כלומר~$S(Σ)⊆❨Σ∪❴.,⟨,⟩❵❩^*.$ גם קבוצת הביטויים הסימבוליים
מעל~$Σ^*$ היא שפה פורמלית מעל אלפאבית זה~$S(Σ^*)⊆❨Σ∪❴.,⟨,⟩❵❩^*$ אבל
כמובן~$S(Σ)⊊ S(Σ^*)$.

לביטויים סימבוליים יש חשיבות רבה בשפת התכנות ליספ. כל תכנית בשפה זו היא ביטוי
סימבולי, המקבלת כקלט ביטוי סימבולי אחד, ומחשבת ממנו ביטוי סימבולי אחר.


