§מבוא
ערכים לעומת טיפוסים.
לימוד שפת תכנות על סמך הערכים הנתמכים.

ליטרלים

§ קבוצת
§ערכים יסודיים מספריים
§§מספרים שלמים
⌘תחילת{ציינון}
• כמה מספרים שלמים ישנם? אינסוף? הבדיחה על אלוהים.
• שיטות יצוג של מספרים שלמים מטווח קטן וסופי.
• שיטות יצוג של מספרים בלתי חסומים.
• רמת הסתרת המכונה: א. הסתרה בלימבו. ב. חשיפת המכונה ג. מכונה אבסטרקטית
כביכול.
• מרחב הטיפוסים השלמים.
⌘סוף{ציינון}
§§מספרים ממשיים
⌘תחילת{ציינון}
• שיטת יצוג נקודה צפה.
• פרדוקס האסוציאטיביות.
• חיזוק מכונה אבסטרקטית כביכול. דיון ב IEEE
• מרחב הטיפוסים הממשיים.
⌘סוף{ציינון}

§§מספרים מרוכבים
מלמד על תכנון השפה.

§ערכים יסודיים אחרים
§§ערכים בוליאניים
⌘תחילת{ציינון}
• הקשר למספרים שלמים, ומספרים בכלל.
• טיפוס נפרד.
• שיטת הכשלון.
⌘סוף{ציינון}

§§תווים
⌘תחילת{ציינון}
• כמה תוים שונים ישנם? שיטות יצוג. הגדרת קבוצת התווים, והצירוף של ביטים
באורך קבוע.UNICODE ועוד.
• הקשר למספרים שלמים. ORD ו SUCC והקשר שלהם לשיטת היצוג.
⌘סוף{ציינון}

§§סדריות
⌘תחילת{ציינון}
• קריטריון השוואה: האם יש חשיפה של שיטת היצוג? השיטה יכולה להיות אורך
ותוכן, מערך המסתיים באפס, מערך של מספרים שלמים? בדרך כלל שיטת היצוג נחשפת: “אסור
שהסדרית תהיה ארוכה מ200 תווים" אסור שהסדרית תכיל את תו האפס, הם רמזים לשיטת
היצוג.
• קריטריון השוואה: הקשר לטיפוס תו.
• קריטריון השוואה: פעולות בשפה? הבעיה: המון פעולות מעניינות ומוכרות, אבל
מוסכמות רק ביחס לשרשור,ולאורך. הסיבה: המון פרטים מורכבים.
⌘סוף{ציינון}
§§ערכים סימבוליים
בדרך כלל יצוג על ידי עצים.
⌘תחילת{ציינון}
• S-EXPRESSIONS
אטוםים ו
NULL
• ערכים בשפת התכנות פרולוג
דומים ל
S-Expressions
אבל יש עוד סוג של ערך פרימטיבי: משתנה.
⌘סוף{ציינון}
§§ מיון טיפוסים יסודיים

⌘תחילת{ציינון}
• שיטת ההגדרה: טיפוסים מובנים, מוגדרים מראש, מוגדרי משתמש
• טיפוסים אטומיים לעומת טיפוסים מורכבים.
• טיפוסים סדורים לעומת טיפוסים לא סדורים.
• טיפוסים רציפים לעומת טיפוסים דיסקרטיים.
⌘סוף{ציינון}

§מערכים

§§סוגי מערכים

⌘תחילת{ציינון}
• סוגי אינדקס
• מערכים דינמיים.
• מערכים אסוציאטיביים
⌘סוף{ציינון}
§§תכנות במערכים בלבד

במובן מסויים מערך הוא הזכרון של המחשב.

לא צריך מצביעים, לא צריך רשומות. אפשר לעשות הכל בעזרת מערכים. כך עשו דברים תמיד
בפורטרן.

תכנות ללא מערכים.

§טיפוסים בלתי שגרתיים

§§הטיפוס Unit

⌘תחילת{ציינון}
• גודל אחסון? אפס!
• שימושי להגדיר פונקציה כפרוצדורה.
• בדרך כלל לא טיפוס שווה ערך. אפשר להגדיר
\LR{void *} אבל אי אפשר להגדיר
\LR{field of type void}
⌘סוף{ציינון}

§§הטיפוס NONE

⌘תחילת{ציינון}
• טוב להגדיר בו פונקציה שאינה חוזרת לעולם.
• אבל, יש מעט פונקציות כאלו \ldots
• מה שבעצם נדרש הוא הגדרה של פונקציה שיכולה להחזיר ערך.
• הערך NaN יכול להחשב ככזה. קרא את המילה NaN כ"כשלון חישובי", וראה
שהסמנטיקה מתאימה. NaN ועוד כל דבר הוא NAN. השוואה של NAN לכל דבר נכשלת,
כולל
לעצמו,

⌘סוף{ציינון}
§§טיפוס של n-יה

⌘תחילת{ציינון}
• זהו תמיד טיפוס מורכב!
• דוגמאה ב⌘פסקל וגם ב ⌘סי.
• ערך מוחזר? דוגמא בשפת גו.
• נוח: לא תמיד קל לתת שמות.
• Unit הוא מקרה של TUPLE באורך אפס. הערך אינו מורכב.
• מדוע אין תמיכה גורפת? א. סמנטיקה. טופלים מקוננים. האם אפשר לקנן את הטפל
הריק? שתי משמעויות. הצבה לתוך טפל שמכיל טפל ריק.

⌘סוף{ציינון}
§טיפוסים מורכבים

§§תיוג TUPLES.

⌘תחילת{ציינון}
• טיפוס רשומה
• הפקודה WITH
• טיפוס מנוי
• יצור הטיפוס Unit והטיפוס NONE באמצעות טיפוס מנוי

⌘סוף{ציינון}
§§ סכום ישר

§§ טיפוס פונקציה

§§ערך מצביע וערך הפנייה

דוגמא יפה בשפת
⌘אלגול
חישוב ערך ההפנייה.
דוגמאות גם בשפת C++

§§טיפוסים רקורסיביים

§§מיכלים

⌘תחילת{ציינון}
•
לכידות השפה
⌘סוף{ציינון}

§תרגילים




